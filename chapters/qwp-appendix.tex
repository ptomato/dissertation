\section{Reciprocity of the slit transmission}
\label{qwp:sec:correction-factor}

\marginnote{This section is an appendix that did not appear in the published paper.}
\sectionstart{We stated earlier in this chapter} that we corrected our model for the finite \gls{NA} of the detector in our experiment.
The necessity of this correction was brought to our attention by an apparent violation of reciprocity in the experiment.
In the experiments described in the foregoing sections, we illuminated the sample on the gold side (hereafter the `forward' configuration), but when we turned the sample around and performed the experiment again while illuminating it from the glass side (the `reverse' configuration), the results were different!

This is, of course, not really a violation of reciprocity, but it is caused by the detector's \gls{NA}.
The exit aperture of the slit is subwavelength, so it radiates in all directions, but not uniformly.
The slit's scattering profile depends on the shape of the mode inside the slit, and also on the medium that the slit scatters into.
Therefore, not all the radiation that actually exits the slit is emitted into the cone of angles that the detector can collect.
Which fraction is collected by the detector depends on the circumstances, meaning that the two configurations cannot be compared directly without correcting for this effect.

We assume that the detector is situated in air, with an index of refraction $n_4$.
In the forward configuration, $n_1 = n_4$ and $n_3$ is the index of the glass substrate.
Conversely, in the reverse configuration, $n_1$ is the index of the glass substrate and $n_3 = n_4$.

\newthought{The slit supports one \gls{TM} and one \gls{TE} mode.}
We treat the slit as a parallel-plate waveguide with metal walls in which we assume that higher-order modes do not propagate.
The metal wall boundary is at $x = \pm w/2$.
We call the complex amplitudes of the waveguide modes inside the slit $\cE_\TM (x, z)$ and $\cE_\TE (x, z)$.
They depend on the permittivity $\epsilon$, the index of the slit material $n_2$, and the slit width $w$.

To find the angular scattering profile of the modes, we take the Fourier transform of the mode profile at the exit aperture of the slit: $\cE_\TM (x, d)$ and $\cE_\TE (x, d)$.
This gives us the scattered electric field amplitude $\cv{F}$ as a function of transverse wavenumber $k_x$.
This is appropriate if the collection objective is in the far field of the slit.
We estimate the Fresnel number $N_F$ using typical values for our experiment,
%
\begin{equation}
N_F = \frac{a^2}{L\lambda} = \frac{w^2}{4L\lambda}
\approx \frac{(5\times 10^{-7})^2}{4 \cdot 1\times 10^{-3} \cdot 8\times 10^{-7}}
\approx \frac{25\times 10^{-14}}{32\times 10^{-10}} \ll 1,
\end{equation}
%
which justifies the assumption of Fraunhofer diffraction at $1\unit{mm}$ distance from the slit. $L$ is in this case the working distance of the objective.

The angle $\theta$ of the corresponding plane wave component is equal to
\begin{equation}\label{qwp:eq:spatial-frequency-angle}
\theta = \arcsin (k_x / n_3 k_0),
\end{equation}
where $k_0 = 2\pi/\lambda$ is the wavenumber in free space.
We plot the angular scattering profiles for a $250\unit{nm}$ wide slit in Fig.~\ref{fig:scattering-profiles}a, calculated numerically by fast Fourier transform.
%
\begin{figure*}[tb]
\centering\includegraphics{graphs/qwp/scattering-profile.pdf}
\caption{(a) Scattering profile of a $250\unit{nm}$ wide slit, as a function of angle.
(b) Transmission of scattered light at the $n_3$--$n_4$ interface as a function of scattering angle.
Note the Brewster angle at the point where the \gls{TM} transmission reaches unity.
Other parameters: $\lambda=800\unit{nm}$, $n_2=1.0$, $n_3=1.5$, $n_4=1.0$.}
\label{fig:scattering-profiles}
\end{figure*}

In the forward configuration (light incident on the air side) described above, we have to take into account the Fresnel losses at the $n_3$--$n_4$ (glass-air) interface.
Part of the scattered energy never leaves the glass substrate, due to total internal reflection.
The \gls{TE} and \gls{TM} components are also transmitted differently, since there is a Brewster angle for \gls{TM}.

For the transmission $T = 1 - R$, we write:
\begin{equation}\label{eq:fresnel-te}
T_\TE(\theta) = 1 - \left(
	\frac{n_3\cos\theta - n_4\sqrt{1 - (\frac{n_3}{n_4}\sin\theta)^2}}
		{n_3\cos\theta + n_4\sqrt{1 - (\frac{n_3}{n_4}\sin\theta)^2}}
\right)^2
\end{equation}
\begin{equation}\label{eq:fresnel-tm}
T_\TM(\theta) = 1 - \left(
	\frac{n_3\sqrt{1 - (\frac{n_3}{n_4}\sin\theta)^2} - n_4\cos\theta}
		{n_3\sqrt{1 - (\frac{n_3}{n_4}\sin\theta)^2} + n_4\cos\theta}
\right)^2
\end{equation}
We plot these transmission profiles in Fig.~\ref{fig:scattering-profiles}b.
They are independent of the slit width, or indeed any of the slit parameters.
Also note that for $n_3=n_4$, $T=1$, as it should be since there is no interface in that case.

The finite \gls{NA} of the detector means that not all of the scattered light is collected.
Light outside a maximum acceptance angle $\theta_\mathrm{max}$ misses the detector.
Due to Snell's law, $\NA=n_3\sin\theta_\mathrm{max}$, and therefore
\begin{equation}
\theta_\mathrm{max} = \arcsin(\NA/n_3),
\end{equation}
no matter what medium $n_4$ the detector is actually in.

The detector signal, then, must be corrected by a factor
\begin{equation}\label{eq:correction}
C_i = \frac{ \displaystyle
	\int_{-\theta_\mathrm{max}}^{\theta_\mathrm{max}}
		T_i(\theta)
		\left| \cv{F}_i (n_3 k_0 \sin\theta) \right|^2
	d\theta
}{ \displaystyle
	\int_{-\pi/2}^{\pi/2}
		\left| \cv{F}_i (n_3 k_0 \sin\theta) \right|^2
	d\theta
},
\end{equation}
where $i\in\{\TE, \TM\}$, which is the ratio of energy entering the detector to the total energy scattered from the slit.
This factor depends on $\lambda$, $\epsilon$, $n_2$, $n_3$, $n_4$, $w$, and $\NA$.
(It does not depend on $d$ or $n_1$; the factor containing $d$ can be taken outside the integrals and divided away, and $n_1$ only influences the coupling to surface plasmon modes on the front interface.)

\newthought{In the experiment described in this chapter,} we take $\lambda$, $\epsilon$, $n_2$, and $n_4$ as design parameters, so we will not examine their influence on the correction factor.
The index $n_3$ can take one of two values: for reverse illumination, $n_3 = n_4$, whereas for forward illumination $n_3 \neq n_4$.
Since we are only interested in the ratio of \gls{TM} to \gls{TE} transmission, we will examine the ratio of the two correction factors $C_\TM/C_\TE$ in four situations: forward and reverse illumination, and for low and high \gls{NA}.

We examine the ratio because the values of $C_\TM$ and $C_\TE$ individually are not particularly surprising; a high-\gls{NA} detector will collect more of the light in both forward and reverse illumination, obviously.
Reverse illumination also causes more light to enter the detector, no matter the \gls{NA} because there is no total internal reflection loss at the $n_3$--$n_4$ interface, so $T_i(\theta) = 1$ in \eqref{eq:correction}.

In Fig.~\ref{qwp:fig:correction-factors}a we plot the ratio $C_\TM/C_\TE$ for reverse illumination, calculated for two different numerical apertures.
We see that the numerical aperture does not influence the shape of the curve very much.
%
\begin{figure*}[tbp]
\centering\includegraphics{graphs/qwp/correction-factors.pdf}
\caption{Ratio of the two \gls{NA} correction factors $C_\TM/C_\TE$: less than 1 means that the detector picks up more \gls{TE} light, more than 1 means the detector picks up more \gls{TM} light.
It is plotted for two different \gls{NA} values: low ($\NA=0.2$, purple), and high ($\NA=0.8$, green).
(a) Reverse illumination ($n_1 = 1.5$, $n_2 = n_3 = n_4 = 1.0$);
(b) forward illumination ($n_1 = n_2 = n_4 = 1.0$, $n_3 = 1.5$).
As usual, we take $\lambda=800\unit{nm}$.}
\label{qwp:fig:correction-factors}
\end{figure*}

Forward illumination, for which the ratio of the correction factors is plotted in Figs.~\ref{qwp:fig:correction-factors}b, is quite different.
If the detector has low \gls{NA}, it behaves much the same as in reverse illumination, with slightly more difference between \gls{TM} and \gls{TE}.
In the high-\gls{NA} case, on the other hand, the detector picks up more \gls{TM} light than \gls{TE} light for almost every slit width: there is about a $6\%$ difference.
A look at Fig.~\ref{fig:scattering-profiles} suggests that this happens when the \gls{NA} is large enough that $\theta_\mathrm{max}$ is large enough that it includes the Brewster angle at the $n_3$--$n_4$ interface, at which all the \gls{TM} light is transmitted and not all the \gls{TE} light.

% \newthought{When performing a transmission} experiment on subwavelength slits of various widths, the model presented here suggests that the measured data must undergo a slight correction depending on the slit width.
% In addition, the correction factor is different for \gls{TE} and \gls{TM} illumination of the slit, due to the difference between the scattering profiles of the lowest-order \gls{TE} and \gls{TM} modes.
% However, even at high numerical aperture, the features in the wings of the scattering profile are not important enough to warrant a substantial correction.

% The correction factor becomes more important when there is a dielectric interface between the exit aperture of the slit and the detector.
% \gls{TE} and \gls{TM}-polarized light are transmitted differently through that interface, due to the existence of Brewster's angle.
% At low numerical aperture, this does not have any influence, but at high numerical aperture, more \gls{TM} light is transmitted through the interface than \gls{TE} light.

% Each of these two modes has its own propagation constant $\beta$, which are the solutions to these two eigenvalue equations:\cite[243]{SnyderLove}
% %
% \begin{equation}
% \frac{W_\TE}{U_\TE} - \tan U_\TE = 0,
% \end{equation}
% %
% \begin{equation}
% \frac{n_2^2 W_\TM}{\epsilon U_\TM} - \tan U_\TM = 0,
% \end{equation}
% %
% in which we use the dimensionless modal parameters $U_i = \frac{w}{2} \sqrt{k_0^2 n_2^2 - \beta_i^2}$, $W_i = \frac{w}{2} \sqrt{\beta_i^2 - k_0^2 \epsilon}$ as a shorthand, where $i \in \{\TE, \TM\}$.\cite[227]{SnyderLove}

% \begin{equation}
% \cE_\TE (x, z) =
% \begin{cases}
%   \exp \left(W_\TE(1 - 2\abs{x}/w) + i\beta_\TE z \right) \uy,
%   & \abs{x} > w/2
%   \\

%   \displaystyle
%   \frac{\cos(2U_\TE x/w) \exp(i \beta_\TE z)}{\cos U_\TE} \uy,
%   & \abs{x} < w/2
% \end{cases}
% \end{equation}
% %
% \begin{equation}
% \cE_\TM (x, z) =
% \begin{cases}
%   \frac{n_2^2}{\epsilon} \exp(W_\TM(1 - 2\abs{x}/w) + i\beta_\TM z) \ux,
%   & \abs{x} > w/2
%   \\

%   \displaystyle
%   \frac{\cos( 2U_\TM x/w) \exp(i \beta_\TM z)}{\cos U_\TM} \ux,
%   & \abs{x} < w/2
% \end{cases}
% \end{equation}

% The effective indices $n_\text{eff} = \beta / k_0$ are plotted in Fig.~\ref{fig:propagation-constants}.

% %
% \begin{figure}[tb]
% \centering\includegraphics{graphs/qwp/effective-index}
% \caption{Real and imaginary parts of the effective indices of refraction for the \gls{TE} and \gls{TM} modes as a function of slit width $w$.
% Other parameters: $\lambda=800\unit{nm}$, $n_2=1.0$.
% }
% \label{fig:propagation-constants}
% \end{figure}

% Example mode profiles at the slit exit for a very thin slit, a medium-width slit, and a slit wider than $\lambda/2$ are shown in Fig.~\ref{fig:waveguide-modes}.
% %
% \begin{figure*}[tb]
% \centering\includegraphics{graphs/qwp/waveguide-modes.pdf}
% \caption{Waveguide modes in three gold slits of different widths.
% (a) $w=50\unit{nm}$, (b) $w=250\unit{nm}$, (c) $w=500\unit{nm}$.
% Other parameters: $\lambda=800\unit{nm}$, $n_2=1.0$.}
% \label{fig:waveguide-modes}
% \end{figure*}