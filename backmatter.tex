\begin{otherlanguage}{dutch}

\nonumberchapter{Samenvatting}

\sectionstart{Een oppervlakteplasmon} is een lichtgolf die gebonden is aan een metaaloppervlak.
Er bestaan twee soorten van: de ene komt voor bij metalen deeltjes met sub-micron afmetingen, de andere bij metaaloppervlakken die tenminste op de schaal van de optische golflengte vlak zijn.
In dit proefschrift, \emph{``Tweedimensionale optica: diffractie en dispersie bij oppervlakteplasmonen,''} gaat het over het laatste type. Deze soort plant zich langs het twee-dimensionale metaaloppervlak voort, in tegenstelling tot `gewoon' licht dat door de driedimensionale ruimte reist.
De binding van oppervlakteplasmonen aan het metaaloppervlak maakt het mogelijk optische signalen te sturen door kanalen met extreem kleine afmetingen.

\section*{Deel 1: Verschijnselen bij dunne spleten in metaalfilms}

\sectionstart{Uit eerder werk,} o.a.\ dat van mijn voorganger Nikolay Kuzmin, was bekend dat onder bepaalde omstandigheden een zeer nauwe spleet of kras in een heel dunne metaallaag invallend licht met de juiste polarisatie om kan zetten in oppervlakteplasmonen, en omgekeerd.

In hoofdstuk~\ref{qwp:chapter} bekijken we bij een metaalfilm gemaakt van goud met een dikte van 0.0002 mm, hoeveel licht door spleten van verschillende breedte, varierend van 0.00005 tot 0.0005 mm, doorgelaten wordt, en hoe dit doorgelaten licht gepolariseerd is.
We gebruiken het aanslaan van oppervlakteplasmonen bij één polarisatie om de polarisatie van het doorgelaten licht te kunnen beheersen.
Het blijkt dat bij een bepaalde spleetbreedte en filmdikte de spleet lineair gepolariseerd licht kan omzetten in circulair gepolariseerd, en omgekeerd.
Wij hebben een simpel model ontwikkeld dat deze uitkomst ook op intuïtieve wijze verklaart.
Dit resultaat is een handige manier om op kleine schaal de functionaliteit van een zogenaamde kwart-lambdaplaat te realiseren.
In hoofdstuk~\ref{soc:chapter} gebruiken we dit verschijnsel nog eens, maar met cirkelvormige spleten, om een optische draaikolk te laten ontstaan uit circulair gepolariseerd licht, en daarmee optisch spinimpulsmoment om te zetten in optisch baanimpulsmoment.

Hoofdstuk~\ref{ch:tomography} beschrijft een proef met twee zeer nauwe spleten die in een heel dunne goudfilm parallel aan elkaar zijn gekerfd.
De ene spleet wordt beschenen met licht; dat wordt daar gedeeltelijk omgezet in oppervlakteplasmonen.
De oppervlakteplasmonen reizen naar de andere spleet waar ze weer omgezet worden in licht.
We meten de lichtverdeling, maar onderweg is deze door buiging van vorm veranderd ten opzichte van de verdeling bij het invallende licht.
Deze vormverandering gebruiken we om informatie over de fase (het golffront) van het invallende licht te achterhalen; de fase kan niet direct worden gemeten en wordt meestal gemeten met behulp van interferentie met een tweede lichtbundel.
Als toepassing van deze techniek meten we de fase van een bundel die een optische draaikolk bevat, maar uiteindelijk kan de techniek leiden tot een golffrontsensor met een veel hogere ruimtelijke resolutie dan de gangbare technieken, wat interessant zou kunnen zijn voor de astronomie en \smallcaps{UV}-lithografie.

\section*{Deel 2: Anomale dispersie van oppervlakteplasmonen}

\sectionstart{Dispersie is het verschijnsel} dat de snelheid waarmee licht zich door een materiaal voortplant afhangt van de golflengte van het licht (de kleur).
Bijvoorbeeld, een puls van rood licht en \'e\'en van blauw die op hetzelfde moment een blok glas worden ingestuurd, komen aan de andere kant op verschillende momenten aan.
Gewoonlijk komt het rode licht eerder aan dan het blauwe (deze situatie wordt `normale dispersie' genoemd), maar soms is het andersom: `anomale' dispersie.
Anomale dispersie heb je nodig om solitonen te laten ontstaan, lichtpulsen die een lange afstand kunnen afleggen zonder te vervormen.

Anomale dispersie komt vaak voor in de buurt van golflengtes die het materiaal absorbeert.
Het metaal aluminium heeft zo’n absorptie in het nabije infrarood.
In het tweede deel van dit werk proberen wij de vraag te beantwoorden of deze absorptie ook anomale dispersie van oppervlakteplasmonen aan een aluminiumoppervlak met zich meebrengt.
Dit onderzoeken wij met een methode, waarbij de oppervlakteplasmonen aangeslagen worden door licht aan te voeren vanuit een prisma.
Deze techniek kent twee varianten, genoemd naar de Duitse onderzoekers Kretschmann en Otto.
Van de Ottoconfiguratie wordt vaak gedacht dat deze alleen maar nadelen biedt vergeleken met de Kretschmannconfiguratie.
In hoofdstuk~\ref{ch:drudium} laten we zien dat dit een misverstand is.
Daarnaast introduceren we een analysemethode waarmee wij de resultaten van experimenten aan verliesgevende metalen met beide opstellingen zinvol kunnen interpreteren, wat niet mogelijk is met de gebruikelijke aanpak.

Hoofdstukken~\ref{ch:kretschmann} en~\ref{ch:otto} beschrijven de meetresultaten aan oppervlakteplasmonen met anomale dispersie.
In hoofdstuk~\ref{ch:kretschmann} tonen we inderdaad anomale dispersie aan bij oppervlakteplasmonen op een aluminiumoppervlak.
Vervolgens maken we de mate van anomale dispersie nog veel groter door het metaal tot vloeibare stikstoftemperatuur af te koelen, ongeveer $-200\unit{^\circ C}$, beschreven in hoofdstuk~\ref{ch:otto}.
Dit is echter een afweging tussen meer anomale dispersie en meer verlies, omdat de oppervlakteplasmonen sneller uitdoven op het gekoelde metaal.

\end{otherlanguage}

\nonumberchapter{Curriculum Vit\ae}

\section*{Philip Francis Chimento \smallcaps{III}}

\begin{fullwidth}
\begin{tabular}{lp{10cm}}
1981       & Born in Raleigh, North Carolina, United States \smallskip \\
1993--1994 &
	{\raggedright Secondary education, Sherwood Githens Middle School, Durham, \\
	North Carolina, United States} \smallskip \\
1994--1999 & Secondary education, \foreignlanguage{dutch}{Het Stedelijk Lyceum}, Enschede, Netherlands \smallskip \\
1999--2008 &
	{\raggedright Bachelor's and Master's degree in applied physics \\
	Twente University, Enschede, Netherlands \\
	\emph{Freshman year completed cum laude}}
	\smallskip \\
2009--2013 &
	{\raggedright PhD research \\
	Leiden Institute of Physics, Leiden University, Leiden, Netherlands}
	\smallskip \\
2013--     & Software engineer, Endless Mobile
\end{tabular}
\end{fullwidth}

\nonumberchapter{List of Publications}
{\raggedright
\begin{itemize}
\item Chimento, P.~F., Jurna, M., Bouwmans, H.~S.~P., Garbacik, E.~T., Hartsuiker, L., Otto, C., Herek, J.~L., \& Offerhaus, H.~L.\ (2009). High-resolution narrowband \smallcaps{CARS} spectroscopy in the spectral fingerprint region. \emph{Journal of Raman Spectroscopy, 40,} 1229--1233.
\item Chimento, P.~F., 't~Hooft, G.~W., \& Eliel, E.~R.\ (2010). Plasmonic optical vortex analyzer. In J. Pozo, M. Mortensen, P. Urbach, X. Leijtens, \& M. Yousefi (Eds.), \emph{Proceedings of the 2010 annual symposium of the \smallcaps{IEEE} Photonics Benelux Chapter,} November 19, 2010 (pp. 17--20). 2010 Annual Symposium of the \smallcaps{IEEE} Photonics Benelux Chapter. Delft, Netherlands: Uitgeverij \smallcaps{TNO}.
\item Chimento, P.~F., 't~Hooft, G.~W., \& Eliel, E.~R.\ (2010). Plasmonic tomography of optical vortices. \emph{Optics Letters, 35,} 3775--3777.
\item Chimento, P.~F., Kuzmin, N.~V., Bosman, J., Alkemade, P.~F.~A., 't~Hooft, G.~W., \& Eliel, E.~R.\ (2011). A subwavelength slit as a quarter-wave retarder. \emph{Optics Express, 19,} 24219--24227.
\item Chimento, P.~F., Alkemade, P.~F.~A., 't~Hooft, G.~W., \& Eliel, E.~R.\ (2012). Optical angular momentum conversion in a nanoslit. \emph{Optics Letters, 37,} 4946--4948.
\item Chimento, P.~F., 't~Hooft, G.~W., \& Eliel, E.~R.\ (2013). When the dip doesn't tell the whole story: interpreting the surface plasmon resonance in lossy metals. Submitted to \emph{Optics Express}.
\item Chimento, P.~F., 't Hooft, G.~W., \& Eliel, E.~R. Anomalous dispersion of surface plasmons. In preparation.
\item Chimento, P.~F., 't Hooft, G.~W., \& Eliel, E.~R. Enhancing the anomalous surface plasmon dispersion in aluminum. In preparation.
\end{itemize}
}

\nonumberchapter{Acknowledgements}

\sectionstart{I would most like to thank} the students that I had the pleasure of mentoring: Carolina Rend\'on Barraza, Johan Bosman, Mark Bogers, David Kok, and Tobias de Jong.
They all contributed in important ways, even though the project that Carolina, David, and Tobias worked on did not make it into the publishable stage because of time constraints.

One is not allowed any more to thank one's coworkers indiscriminately, but some people deserve a mention for their contributions beyond those of the co-authors on my papers.
Wolfgang L\"offler's expertise is woven all throughout this book; he was always ready to bounce ideas off and share lab tips.
Michiel de Dood took a special interest in the aluminum project (chapters~\ref{ch:kretschmann} and~\ref{ch:otto}) and our discussions were invaluable in understanding the solid-state physics involved.
Daan Boltje put time into preparing the Kretschmann prisms used in chapter~\ref{ch:kretschmann}.

The work described in chapter~\ref{ch:otto} involved cryostats and liquid nitrogen, something I had had little experience with when I started.
Jelmer Renema helped to close this experience gap, and assisted with the \smallcaps{COMSOL} heat flow simulations.
Mirthe Bergman, Arjen Geluk, and others in the Fine Mechanics Department worked on the cryostat that I used and made sure it was simple, easy, and leak-free.

Philippe Lalanne, professor at the Institut d'Optique, \smallcaps{CNRS}, was willing to share the Gaussian quadrature code from their paper\cite{Lalanne2006} which I adapted for chapter~\ref{qwp:chapter}.
Speaking of sharing computer code, I relied heavily on open source software almost from the start of this research.
NumPy and SciPy\cite{Jones2001} did all the number crunching.
I made all the graphs in this book with Matplotlib\cite{Hunter2007} and the diagrams with Inkscape.
I used DataThief \smallcaps{III}\cite{Tummers2006} to digitize printed specs of anti-reflection coatings.

% Bring number of pages to a multiple of 4
\newpage
\thispagestyle{empty}
\mbox{}
\newpage
\thispagestyle{empty}
\mbox{}
\newpage
\thispagestyle{empty}
\mbox{}
