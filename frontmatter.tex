% Plain title page, recto 1
\maketitle

% Copyright page, verso 2
\clearpage\thispagestyle{empty}
\begin{fullwidth}
	~\vfill

	\setlength{\parindent}{0pt}
	\setlength{\parskip}{\baselineskip}

	% \begin{otherlanguage}{dutch}
	% 	\raggedright\noindent\sffamily
	% 	\MakeTextUppercase{\textbf{Tweedimensionale optica:} diffractie en dispersie bij oppervlakteplasmonen}
	% \end{otherlanguage}
	% \vspace{1cm}

	Copyright \copyright\ \the\year\ \thanklessauthor

	Published by \thanklesspublisher

	Cover design: \thanklessauthor

	Printed by Gildeprint Drukkerijen, The Netherlands

	\smallcaps{ISBN:} 978-90-8593-155-3
\end{fullwidth}

% Full title page, recto 3
\clearpage\thispagestyle{empty}
\begin{fullwidth}
\begin{center}
	\Large
	{\par\sffamily\fontsize{28}{32}\selectfont\par
	TWO-DIMENSIONAL OPTICS}

	{\par\sffamily\fontsize{18}{22}\selectfont\par
	DIFFRACTION AND DISPERSION\\ OF SURFACE PLASMONS}

	\vfill
	Proefschrift

	\vspace{2\baselineskip}
	ter verkrijging van

	de graad van Doctor aan de Universiteit Leiden

	op gezag van Rector Magnificus prof.mr. C.J.J.M. Stolker,

	volgens besluit van het College voor Promoties

	te verdedigen op woensdag 22 mei 2013

	klokke 11:15 uur

	\vspace{2\baselineskip}
	door

	\vspace{\baselineskip}
	Philip Chimento

	geboren te Raleigh, North Carolina, Verenigde Staten

	in 1981
\end{center}
\end{fullwidth}
% \begin{fullwidth}
% 	\sffamily
% 	\fontsize{18}{20}\selectfont\par\noindent\textcolor{darkgray}{\allcaps{\thanklessauthor}}
% 	\vspace{11.5pc}
% 	\fontsize{36}{40}\selectfont\par\noindent\textcolor{darkgray}{\allcaps{\thanklesstitle}}
% \end{fullwidth}
% \vfill

% \begin{otherlanguage}{dutch}
% 	\raggedright
% 	% Temporarily change \parskip
% 	\setlength{\parskip}{1em}
% 	\setlength{\parindent}{0em}
% 	\newthought{Proefschrift} ter verkrijging van de graad van doctor aan de Universiteit Leiden, op gezag van rector magnificus prof.~mr.~C.~J.~J.~M.\ Stolker, volgens besluit van het college voor promoties te verdedigen op
% 	\makebox[2em]{\hrulefill}dag
% 	\makebox[6em]{\hrulefill} 2013 klokke
% 	\makebox[3em]{\hrulefill} uur

% 	\newthought{Door} ir.\ Philip Francis Chimento \textsc{iii}, geboren te Raleigh, North Carolina, Verenigde Staten in 1981
% \end{otherlanguage}

% Promotiecommissie, verso 4
\clearpage\thispagestyle{plain}
\begin{fullwidth}
\begin{otherlanguage}{dutch}
\noindent\textsf{\fontsize{18}{20}\selectfont\allcaps{Promotiecommissie}}\vspace{5mm}

\noindent
\begin{tabular}{lll}
Promotores:
	& Prof.~dr.~E.~R.~Eliel & Universiteit Leiden \\
	& Prof.~dr.~G.~W.~'t~Hooft & Philips Research \& Universiteit Leiden \\
Leden:
	& Dr.~M.~P.~van Exter & Universiteit Leiden \\
	& Dr.~C.~Genet & Universit\'e de Strasbourg \\
	& Prof.~dr.~J.~G\'omez Rivas & Philips Research \& \smallcaps{TU} Eindhoven \\
	& Dr.~H.~L.~Offerhaus & Universiteit Twente \\
	& Prof.~dr.~T.~D.~Visser & Vrije Universiteit \& \smallcaps{TU} Delft \\
	& Prof.~dr.~J.~P.~Woerdman & Universiteit Leiden
\end{tabular}
\end{otherlanguage}
\vfill


{\parbox{\textwidth}{\raggedright
% Temporarily change \parskip
\setlength{\parskip}{1em}
\setlength{\parindent}{0em}
This work is part of the research program of the Foundation for Fundamental Research on Matter (\smallcaps{FOM}), which is part of the Netherlands Organization for Scientific Research (\smallcaps{NWO}).

An electronic version of this disseration is available at the Leiden University Repository (\url{https://openaccess.leidenuniv.nl}).

Casimir PhD series, Delft-Leiden, 2013-14
}}
\end{fullwidth}

% Dedication and Quotes, recto 5
\cleardoublepage\thispagestyle{plain}

~\vfill

\begin{fullwidth}
\begin{flushright}\itshape
To caffeine

To the future, let's hope it's worth saving

% To my parents, Mahatma Gandhi and Ayn Rand, for teaching me punctuation

% To the noble kangaroo

% Dedicated to the hardest thing I ever did --- this was only the second hardest.

% To Halbe ``I don't know much about science, but I know what I like'' Zijlstra

% To House
\end{flushright}
\end{fullwidth}

\vfill
\vfill

% \begin{fullwidth}
% \begin{flushright}
% \begin{minipage}{10.2cm}
% \begin{quote}
% % \foreignlanguage{german}{Kretschmann und Raether fanden beim Aufbringen einer rauhen Silberschicht in der [...]
% % angegebenen Geometrie eine nach der Vakuumseite austretende Streustrahlung, die sie als Abstrahlung gestreuter Oberfl\"achenwellen deuten.}
% Kretschmann and Raether, applying a rough silver layer in the [...] indicated geometry, found scattered radiation exiting from the vacuum side, which they interpreted as emission of scattered surface plasmons.
% \signed{Otto, 1969}
% \end{quote}
% % \attribution{Andreas Otto, pooh-poohing the Kretschmann configuration by insinuating that Kretschmann couldn't tell rough from smooth}
% \nocite{Otto1969}

% \begin{quote}
% Otto suggested this method of exciting surface plasmons using a prism.
% He does not, however, bring the metal layer into contact with the prism, but instead introduces a thin air layer between the prism and the metal.
% His arrangement is less advantageous for determining optical constants.
% It is experimentally laborious, and a homogeneous air layer thickness, essential for the analysis, is very difficult to achieve.
% Remarkably, Otto explicitly rules out our arrangement described here.
% The figure [...], in which he calculates the reflection from a silver layer under total internal reflection from a prism, is [...] wrong.
% % \foreignlanguage{german}{Diese Methode, mit Hilfe eines Prismas Oberfl\"achenwellen anzuregen, wurde von Otto vorgeschlagen.
% % Er bringt die Metallschicht aber nicht in direkten Kontakt auf das Prisma, sondern sieht eine d\"unne Luftschicht zwischen Prisma und Metall vor.
% % Seine Anordnung ist f\"ur die Bestimmung von optischen Konstanten weniger g\"unstig.
% % Sie ist experimentell aufwendiger und eine f\"ur die Auswertung notwendige homogene Dicke der Luftschicht ist sehr schwer zu realisieren.
% % Merkw\"urdigerweise schlie\ss{}t Otto die von uns hier angegebene Anordnung ausdr\"ucklich aus.
% % Seine Figur [...], in der er die Reflexion an einer Silberschicht bei Totalreflexion an einem Prisma berechnet hat, ist [...] nicht richtig.}
% \signed{Kretschmann, 1971}
% \end{quote}
% % \attribution{Erwin Kretschmann, pooh-poohing the Otto configuration with a straw-man argument about air layers}
% \nocite{Kretschmann1971}

% \begin{quote}
% The objections to this device are due to an error.
% \signed{Raether, 1988, p.\ 11}
% \end{quote}
% % \attribution{Heinz Raether, weighing in more tactfully}
% \nocite{Raether}
% \end{minipage}
% \end{flushright}
% \end{fullwidth}

% Alternate dedication, recto 5
% \includegraphics[width=65mm]{giotto-st-francis-stigmata.jpg}
% \vspace{5mm}

% {\parbox{9.5cm}{\itshape\raggedleft
% Dedicated to Laser Jesus, who invented the laser in 1224 \textsc{ce}, although Ted Maiman is usually credited.
% Above is Giotto di~Bondone's masterful illustration of Laser Jesus' groundbreaking experiment, which unfortunately severely injured Francis of Assisi, thereby demonstrating the importance of proper laboratory safety.}}
% \vfill
% {\noindent\small\href{http://commons.wikimedia.org/wiki/File%3AGiotto_di_Bondone_002.jpg}{Giotto [Public domain], via Wikimedia Commons}}
% \end{flushright}
% \end{fullwidth}

% Table of contents, recto 7
\pagestyle{plain}
\begin{fullwidth}
\tableofcontents
\end{fullwidth}
